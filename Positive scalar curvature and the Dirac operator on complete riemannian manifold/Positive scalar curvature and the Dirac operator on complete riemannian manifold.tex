\documentclass[UTF8]{article}[a4paper,12pt]
\usepackage[thmmarks]{ntheorem}
\usepackage{amsmath}
\usepackage{amsfonts,amssymb} 
\usepackage{thmtools}
\usepackage[hmargin=2.5cm,vmargin=2.5cm]{geometry}
\usepackage{tikz-cd,tikz}
\usepackage{graphicx,float}
\usepackage{fancyhdr}
\usepackage{fourier-orns}
\usepackage{quiver}
\setcounter{section}{-1}

%定义命令
\def\N{\mathbb{N}}
\def\Z{\mathbb{Z}}
\def\Q{\mathbb{Q}}
\def\R{\mathbb{R}}
\def\C{\mathbb{C}}
\def\S{\mathbb{S}}
\def\D{\mathbb{D}}
\def\H{\mathbb{H}}

%页眉设计
\renewcommand 
\headrule{
\hrulefill
\raisebox{-2.1pt}
{\quad{\FourierOrns M T S N}\quad}
\hrulefill}
\pagestyle{fancy}

%超链接红色
\usepackage[colorlinks,linkcolor=red]{hyperref}

\usepackage{enumerate}


\title{Positive scalar curvature and the Dirac operator on complete riemannian manifold}
\author{Notes Author:Tsechi/Tseyu}
\begin{document}
\maketitle
\tableofcontents
\section{Some prerequisites of this note}
\subsection{Schwartz kernel}

\section{Generalized Dirac Operators on a Complete Manifold}

Let $\mathrm{Cl}(X)$ denote the \textbf{Clifford bundle} of X. This is the bundle over $X$ whose fiber at a point $x \in X$ is the Clifford algebra $\mathrm{Cl}(T_x X)$.For the definition of Clifford algebra, we recommand the  \href{https://en.wikipedia.org/wiki/Clifford_algebra}{website}.

Furthermore, the riemannian metric and connection extend to $\mathrm{Cl}(X)$ with the properties that:covariant differentiation $\nabla$ perserves the metric, and:
\begin{equation}
    \nabla(\varphi \cdot \psi)=(\nabla \varphi)\cdot \psi+\varphi \cdot (\nabla \psi)
\end{equation}
for all sections $\varphi,\psi \in \Gamma(\mathrm{Cl}(X))$.


\end{document}